\documentclass[twocolumn]{article}

\usepackage[top=0.5in, bottom=0.5in, left=0.5in, right=0.5in]{geometry}

\usepackage{amsmath}
\usepackage{authblk}
\usepackage{graphicx}
\usepackage{natbib}

\newcommand{\cs}[1]{\textbf{CS: {#1}}}
\newcommand{\af}[1]{\textbf{AF: {#1}}}

\newcommand{\Lya}{Lyman-$\alpha$}
\newcommand{\Lyaf}{Lyman-$\alpha$ forest}
\newcommand{\mf}{\langle F \rangle}
\newcommand{\hinv}{h^{-1}}

\renewcommand{\vec}[1]{\mathbf{#1}}

\newcommand{\xx}{\vec x}
\newcommand{\kk}{\vec k}


\begin{document}

\title{
    Notes on the Lyman-$\alpha$ Forest 1D Flux Power Spectrum
}

\author[1]{Casey W. Stark}
\author[2]{Andreu Font}
\affil[1]{Department of Astronomy, University of California, Berkeley, CA
94720, USA}
\affil[2]{Lawrence Berkeley National Laboratory, 1 Cyclotron Road, Berkeley, CA
93720, USA}

\maketitle

\section{Approaches}

Let's discuss all the various approaches to modeling the \Lyaf\ and the 1D power
specifically. We will focus on results from hydro simulations, but other
approaches might be more appropriate for quick tests, and we should mention more
advanced possibilities.

\begin{itemize}
  \item Seljak 2012?
  \item lognormal realizations
  \item N-body sim + FGPA
  \item HPM sim + FGPA
  \item Hydro sim
  \item Hydro sim + RT post-processing
  \item Rad-hydro sim
\end{itemize}

\section{Misc. Background}

This section is just a repository for the notation, conventions, and other
things I use frequently but still forget sometimes.

Fourier convention: $x$ is comoving scale and $k = 2 \pi / x$ is the
corresponding Fourier mode. $\hat{\delta}$ is the Fourier transform of $\delta$.
$V$ is the periodic volume over which we sample $\delta$.

\begin{equation}
  \begin{aligned}
    & \hat \delta(\kk) = \frac{1}{V} \int \delta(\xx) e^{i \kk \cdot \xx} d^3x \\
    & \delta(\xx) = \frac{V}{(2 \pi)^3} \int \hat \delta(\kk) e^{-i \kk \cdot \xx} d^3k
  \end{aligned}
\end{equation}

The convention with no $V$ factors is more common, but this one has the
advantage that $\delta$ is dimensionless in real and Fourier space. The power
spectrum is defined as

\begin{equation}
  P(k) = \langle P(\kk) \rangle = V \langle \hat{\delta} \hat{\delta}^* \rangle
\end{equation}

where the average $\langle \rangle$ is over shells in $k$-space. Note that the
power spectrum has units of volume. Sometimes we consider the dimensionless
power spectrum instead, defined as $\Delta^2(k) = \frac{d \sigma^2}{d \log k}$
where $\sigma^2 = \frac{1}{(2 \pi)^3} \int P(k) d^3k$ is the mass variance, so
that in 3D, $\Delta^2(k) = \frac{k^3 P(k)}{2 \pi^2}$.

1D stuff...

\begin{equation}
  \begin{aligned}
    & \hat{\delta}_{\rm 1D}(k_\parallel)
      = L^{-1} \int \delta(x_\parallel)
        e^{i k_\parallel x_\parallel} dx_\parallel \\
    & \Delta^2_{\rm 1D}(k_\parallel)
      = \frac{k_\parallel}{\pi} P_{\rm 1D}(k_\parallel)
      = \frac{k_\parallel}{\pi} L
        \langle \hat{\delta}_{\rm 1D} \hat{\delta}_{\rm 1D}^* \rangle \\
  \end{aligned}
\end{equation}
where $L$ is the sightline length and the average $\langle \rangle$ is over
modes with magnitude $k_\parallel$.

Relation between various cosmological coordinates (at a fixed redshift) is
$dr = c dt = a dx = \frac{dv}{H}$

Flux perturbations are $\delta_F = F / \mf - 1$, where the mean flux is
the average over all sightlines at that redshift.



\bibliographystyle{plain}
\bibliography{notes}

\end{document}
